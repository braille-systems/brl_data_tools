\documentclass{main.tex}[subfiles]
\begin{document}
\newpage
\section{Собранные данные и разработанный программный продукт}
\subsection{Полученный набор данных}

Для исследования выбраны два произведения на английском языке, которые доступны в фондах Санкт-Петербургской государственной библиотеки для слепых и слабовидящих: "Jane Eyre"\hspace{0pt} (автор -- Шарлотта Бронте; 1196 страниц, 8 брайлевских томов) и "The Scarlet Letter"\hspace{0pt} (автор -- Натаниэль Готорн; 538 брайлевских страниц, 4 тома).
Обе книги напечатаны двусторонним способом, использован ключ Брайля уровня 1 (Grade 1 Braille), т. е. сокращения не использованы, что упрощает предобработку текста.

Съёмка производилась при дневном свете без штатива (с рук).
Книга размещалась на столе, который расположен перед окном, верхней частью листа к окну; лист целиком попадает в поле зрения камеры.
Использованы следующие устройства:
% TODO write about generalization, representativeness...

\begin{enumerate}[noitemsep]
    \item Фотоаппарат Canon PowerShot A800 (10.0 МПикс) -- 1 том (Scarlet Letter, т. 3)
    \item Планшетный ПК IPad Air модели A1475 (1.2 МПикс) -- 6 томов (Jane Eyre т. 1-4, Scarlet Letter т. 1-2)
    \item Смартфон Sony XPeria Z3 Compact (20.7 МПикс) -- 5 томов (Jane Eyre т. 5-8, Scarlet Letter т. 4)
\end{enumerate}
Вспышка не была применена.
После съёмки фотографии были при необходимости обрезаны так, чтобы лист занимал большую часть изображения; прочая предобработка (фильтрация шумов, повышение яркости и т. д.) не проводилась.

Стоит отметить: получение качественных фотографий было облегчено тем, что книги оформлены в переплёте с пластмассовыми кольцами и несложно было раскрыть нужный разворот, почти не изгибая страницы.
Результат распознавания книг в обычном переплёте может быть хуже.

\subsection{Программный продукт}

Алгоритм, работающий с собранными данными, реализован в виде набора сценариев, написанных на языке Python3 и оформлен в репозитории\footnote{\href{https://github.com/braille-systems/brl_data_tools}{github.com/braille-systems/brl\_data\_tools}}.
Программа Angelina Reader была разработана в первую очередь для работы с русскими текстами, поэтому она была модифицирована для правильной работы с английской азбукой Брайля и включена в репозиторий как подмодуль.

\end{document}