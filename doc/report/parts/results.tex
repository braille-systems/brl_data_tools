\documentclass{main.tex}[subfiles]
\begin{document}
\newpage
\section{Выводы}
% TODO remove this in final report

По результатам применения разработанных алгоритмов к собранным данным можно отметить следующее:
\begin{enumerate}[noitemsep]
	\item  Аннотации к неразмеченным данным, полученные при распознавании страниц нейронной сетью, обученной на размеченных данных, достаточно хорошие, чтобы успешно найти соответствующий регион в исходном тексте документа и построить соответствие (выравнивание).
	\item Как видно из рис. \ref{fig:frequencies}, качество меток после коррекции значительно улучшается: доля слов, которые есть в словаре, существенно возрастает (медианное значение в применении к собранным данным составляет 1.0 после исправления и 0.84 до исправления).
	\item При обучении нейронной сети на выборке, в которую включены данные с исправленными псевдометками, наблюдается улучшение в качестве распознавания текстов, написанных английским шрифтом Брайля (табл. \ref{table:validate_results}).
	Это показывает эффективность предложенной схемы.
	В частности, неразмеченные страницы с английским шрифтом Брайля содержат символ <<признак заглавной буквы латинского алфавита>>, который отсутствует в размеченном наборе.
	При генерации псевдометок эти знаки были отмечены как неизвестные; после выравнивания метки были изменены на правильные, и после нового цикла тренировки нейронная сеть научилась их распознавать.

	Впрочем, качество распознавания валидационных наборов, взятых из AngelinaDataset и DSBI, не улучшается или даже немного ухудшается.
	Заметное ухудшение качества рукописных текстов (а также сканов из DSBI) объясняется тем, что в добавленных данных нет рукописей и сканов, следовательно, теперь они реже встречаются среди обучающих примеров.
    % TODO mention misplaced labels problem
	К сожалению, к рукописям описанный в данной работе алгоритм в текущей реализации неприменим: даже если исходный текст известен, нельзя исключать вероятность ошибки, допущенной при письме.
\end{enumerate}

\end{document}
