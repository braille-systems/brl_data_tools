\documentclass{main.tex}[subfiles]
\begin{document}
\subsection{Результаты полуконтролируемого обучения}
По итогам применения реализованных алгоритмов к собранным данным получены следующие результаты:
\begin{enumerate}[noitemsep]
	\item  Аннотации к неразмеченным данным, полученные при распознавании страниц нейронной сетью, обученной на размеченных данных, достаточно хорошие, чтобы успешно найти соответствующий регион в исходном тексте документа и построить соответствие (выравнивание).
	\item Как видно из рис. \ref{fig:frequencies}, качество меток после коррекции значительно улучшается: доля слов, которые есть в словаре, существенно возрастает (среднее значение в применении к собранным данным составляет 0.995 после исправления и 0.838 до исправления).
	\item При обучении нейронной сети на выборке, в которую включены данные с исправленными псевдометками, наблюдается улучшение в качестве распознавания текстов, написанных английским шрифтом Брайля (табл. \ref{table:validate_results}).
	Это показывает эффективность предложенной схемы.
	В частности, неразмеченные страницы с английским шрифтом Брайля содержат символ <<признак заглавной буквы латинского алфавита>>, который отсутствует в размеченном наборе.
	При генерации псевдометок эти знаки были отмечены как неизвестные; после выравнивания метки были изменены на правильные, и после нового цикла тренировки нейронная сеть научилась их распознавать.

	Впрочем, качество распознавания валидационных наборов, взятых из An\-ge\-li\-na\-Da\-ta\-set и DSBI, не улучшается или даже немного ухудшается.
	Заметное ухудшение качества рукописных текстов (а также сканов из DSBI) объясняется тем, что в добавленных данных нет рукописей и сканов, следовательно, теперь они реже встречаются среди обучающих примеров.

    К сожалению, к рукописям описанный в данной работе алгоритм в текущей реализации неприменим: даже если исходный текст известен, нельзя исключать вероятность ошибки, допущенной при письме.
    Также остаются нерешёнными проблемы выпадения символов (рис. \ref{fig:recogn_mistakes:del}) и неверно определённых границ символа (рис. \ref{fig:recogn_mistakes:misplaced_label}); помимо того, нельзя исключать возможные неправильные замены при выравнивании и полиграфические ошибки при печати документов шрифтом Брайля, которые приводят к ухудшению качества меток.
\end{enumerate}

\subsection{Анализ типичных ошибок распознавания}

Помимо исправления псевдометок для полуконтролируемого обучения, выравнивания могут быть использованы для анализа ошибок, допускаемых алгоритмом.
К примеру, в описанном выше отфильтрованном наборе выравниваний (суммарно 150 619 слов) были найдены следующие закономерности:
\begin{enumerate}[noitemsep]
    \item Большинство слов (111 715) распознаны без ошибок вместе с примыкающими к ним знаками препинания.
    \item Ошибки в распознавании слов распределены так: 16899 -- в пунктуации или специальных символах (например, в распознавании цифрового знака или признака заглавной буквы), 18111 -- ошибки в распознавании букв, которые порождают слова, не встречающиеся в словаре; в 3894 словах были допущены замены или пропуски, которые превратили их в другие слова из словаря (например, "whither"\hspace{0pt} $\rightarrow$ "whether"\hspace{0pt}).
    Найдены некоторые характерные замены/несовпадения подобного рода, например, "whom"\hspace{0pt} $\rightarrow$ "whop"\hspace{0pt} (10 раз), "they"\hspace{0pt} $\rightarrow$ "the"\hspace{0pt} (45 раз), "too"\hspace{0pt} $\rightarrow$ "tor"\hspace{0pt} (20 раз).

    Помимо того, в 1014 случаях ошибка заключалась в пропущенном пробеле (два слова слиты в одно) и в 5385 случаях -- в добавлении лишнего пробела.
    \item Как показал подсчёт различных пар знаков в выравнивании (см. рис. \ref{fig:heatmap_alphabet}, рис. \ref{fig:heatmap_special} и табл. \ref{table:per_letter_errs} в приложении), многие символы появляются в результате ошибочного распознавания какого-то конкретного символа значительно чаще, чем в результате замен других (как правило, верный знак отличается от неверного одной точкой из шести).
    Например, ошибочно определённая буква "o"\hspace{0pt} (точки \textnumero 1, 3, 5) в $92\%$ всех случаев получается в результате неверного распознавания буквы "r"\hspace{0pt} (точки \textnumero 1, 2, 3, 5), т. е. из-за пропущенной точки 2.
\end{enumerate}


\end{document}
