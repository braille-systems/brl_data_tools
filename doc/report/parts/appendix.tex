\documentclass{main.tex}[subfiles]
\begin{document}
\newpage
\nonumsection{Приложение 1. Иллюстрации}

\begin{figure}[H]
    \centering
    \begin{subfigure}{.5\textwidth}
        \centering
        \includegraphics[width=\myPictWidth]{test_find_regions/find_regions_k2}
        \caption{$ k = 2 $}
        % TODO \label{fig:}
    \end{subfigure}%
    \begin{subfigure}{.5\textwidth}
        \centering
        \includegraphics[width=\myPictWidth]{test_find_regions/find_regions_k4}
        \caption{$ k = 4 $}
        % TODO \label{fig:}
    \end{subfigure}

    \begin{subfigure}{.5\textwidth}
        \centering
        \includegraphics[width=\myPictWidth]{test_find_regions/find_regions_k8}
        \caption{$ k = 8 $}
        % TODO \label{fig:}
    \end{subfigure}%
    \begin{subfigure}{.5\textwidth}
        \centering
        \includegraphics[width=\myPictWidth]{test_find_regions/find_regions_k15}
        \caption{$ k = 15 $}
        % TODO \label{fig:}
    \end{subfigure}
    \caption{Зависимость числа k-грам, найденных в регионе исходного текста, от позиции региона относительно начала текста}
    \label{fig:find_k}
\end{figure}

% TODO examples of mistakes

\newpage
\nonumsection{Приложение 2. Исходный код программ}
Ниже приведён исходный код программы, реализующей модифицированный алгоритм Нидлмана-Вунша. % TODO ссылка на номер алгоритма
Этот код, а также программы, осуществляющие поиск регионов интереса, вспомогательные манипуляции с изображением (обрезка, переименование...) и текстом (замена символов, подсчёт статистик...) доступны в репозитории\footnote{\href{https://github.com/braille-systems/brl_data_tools}{github.com/braille-systems/brl\_data\_tools}}.
Там же можно найти инструкцию по установке и запуску, а также сслыки для загрузки файлов с исходными данными (файл \texttt{README.md}).

\lstinputlisting{../../scripts/needleman_wunsch.py}
\end{document}