\documentclass{main.tex}[subfiles]
\begin{document}
\thispagestyle{empty}
\section*{РЕФЕРАТ}

На \pageref*{LastPage} стр., 8 рисунков, 2 таблицы.

КЛЮЧЕВЫЕ СЛОВА: ОПТИЧЕСКОЕ РАСПОЗНАВАНИЕ СИМВОЛОВ, СВЁРТОЧНАЯ НЕЙРОННАЯ СЕТЬ, ПОЛУКОНТРОЛИРУЕМОЕ МАШИННОЕ ОБУЧЕНИЕ, ВЫРАВНИВАНИЕ ПОСЛЕДОВАТЕЛЬНОСТЕЙ, ШРИФТ БРАЙЛЯ.

Тема работы -- <<Применение полуконтролируемого машинного обучения к задаче обнаружения символов Брайля на фотоснимках>>.

В работе рассмотрена задача распознавания символов шрифта Брайля на изображениях; предложен способ улучшения качества работы алгоритмов машинного обучения, решающих эту задачу, основанный на использовании больших объёмов фотографий без аннотаций (т.е. на которых неизвестно расположение брайлевских букв).

Задачи, решённые в рамках исследования:
\begin{enumerate}[noitemsep]
    \item Изучение проблемы распознавания шрифта Брайля, её особенностей по сравнению с распознаванием обычного текста; существующих методов решения и наборов тренировочных данных.
    \item Разработка схемы улучшения алгоритма машинного обучения, уже обученного на данных с аннотациями, с помощью данных без аннотаций.
    \item Сбор данных -- фотографий текста, напечатанного шрифтом Брайля.
    \item Применение разработанной схемы и полученных данных к одному из известных ранее алгоритмов распознавания, а также оценка эффективности этой схемы.
\end{enumerate}

В работе используется и дорабатывается программа распознавания Angelina Braille Reader, написанная на языке Python3 и основанная на нейронной сети Retina\-Net (изначально обученной на размеченном наборе из 300 фотографий).
Разработан и реализован на языке Python3 алгоритм исправления меток, полученных в результате распознавания печатных изданий, с помощью выравнивания распознанного текста и исходного электронного текста издания.
Получен набор данных (1731 фотография), к которому применён созданный алгоритм.
Книги, фотографии которых вошли в созданный набор данных, предоставлены Государственной библиотекой для слепых и слабовидящих г. Санкт-Петербурга.
Методическое консультирование оказывали преподаватели СПб ГБУ <<Центр медико-социальной реабилитации инвалидов по зрению>>.

Как показал эксперимент, исправление меток позволяет дообучить нейронную сеть на новых данных, улучшив распознавание печатного текста.

Спроектированный метод может быть применён к любому алгоритму распознавания шрифта Брайля, основанному на машинном обучении.
Производимые при этом выравнивания текстов позволяют провести статистический анализ ошибок распознавания, необходимый для разработки алгоритмов постобработки и исправления распознанных знаков.

\newpage
\thispagestyle{empty}
\section*{ABSTRACT}

\pageref*{LastPage} pp., 8 figures, 2 tables.

KEYWORDS: OPTICAL CHARACTER RECOGNITION, CONVOLUTIONAL\\
\noindent NEU\-RAL NETWORK, SEMI-SUPERVISED MACHINE LEARNING, SE\-QUENCE ALIGN\-MENT, BRAILLE.

The subject of this work is "Semi-supervised learning for optical Braille recog\-ni\-ti\-on".

This study explores the problem of image-based Braille recognition; a data-driven approach is proposed to improve existing machine learning algorithms that solve this problem.
The aforementioned approach utilizes large amounts of unlabeled pictures with printed Braille and their source texts, aligning the original and recognized text.

During the elaboration of the problem, the following tasks were completed:
\begin{enumerate}[noitemsep]
    \item Studying the problem of optical Braille recognition, its differences from con\-ven\-ti\-o\-nal optical character recognition, existing approaches and data sets.
    \item Development of a scheme for improving a machine learning algorithm, pre-trained on labeled data, using unlabeled data.
    \item Collecting photos of text printed in Braille.
    \item Application of the developed scheme along with collected data to one of the existing recognition algorithms; scheme evaluation.
\end{enumerate}

In this work, a program for optical Braille recognition named "Angelina Braille Reader"\hspace{0pt}, which is written in Python3, based on Retina\-Net architecture and initially trained on 300 labeled pictures, is used and modified.
An algorithm for correcting labels based on alignment of recognized and original text is developed and implemented in Python3.
A data set of 1,731 photos has been created, to which the algorithm has been applied.
Books, photos of which are included in the data set, were provided by the St. Petersburg State library for the Blind.
Methodological advice was provided by the teachers of the St. Petersburg State Budgetary Institution "Center for Medical and Social Rehabilitation of the Visually Impaired".

As demonstrated experimentally, correcting labels allows additional training of the neural net on new data which leads to an improvement in the recognition of printed Braille.

The developed technique may be applied to any optical Braille recognition algorithm based on machine learning.
The alignments made in this process allow a statistical analysis of algorithm errors to be performed, which is necessary for the development of post-processing algorithms and the correction of recognized characters.

\newpage
\end{document}