\documentclass{main.tex}[subfiles]
\begin{document}
\section{Введение}
По оценкам Всемирной организации здравоохранения, в мире более % TODO
незрячих, и их число возрастает. % TODO ссылка
Начиная с XIX века лица с ограничениями по зрению используют для чтения и письма рельефно-точечный шрифт Л. Брайля; позже появились особые печатные машинки, а затем и рельефные дисплеи, поддерживающие алфавит Брайля.
В настоящее время обучение технике чтения и письма рельефно-точечным шрифтом входит в программу реабилитации инвалидов по зрению во всём мире.

Ожидается, что развитию и повышению удобства использования азбуки Брайля может серьёзно способствовать появление программ, распознающих рельефный шрифт по фотографии.
Подобные программы могут применяться, например, в следующих ситуациях:
\begin{enumerate}[noitemsep]
	\item % TODO
\end{enumerate}

\cite{begun2019}
\cite{zuev2021}
\end{document}