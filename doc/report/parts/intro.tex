\documentclass{main.tex}[subfiles]
\begin{document}
\nonumsection{Введение}
По оценкам Всемирной организации здравоохранения, в мире более 285 миллионов незрячих или слабовидящих, и их число возрастает \cite{who_report2012}.
В XIX веке люди с ограничениями по зрению начали использовать для чтения и письма рельефно-точечный шрифт Л. Брайля; позже появились особые печатные машинки, а затем и рельефные дисплеи, поддерживающие алфавит Брайля.
В настоящее время обучение технике чтения и письма рельефно-точечным шрифтом входит в программу реабилитации инвалидов по зрению во всём мире.

Ожидается, что развитию и повышению удобства использования азбуки Брайля будет способствовать появление программ, распознающих рельефный шрифт по фотографии.
Подобные программы могут применяться, например, в следующих ситуациях:
\begin{enumerate}[noitemsep]
    \item Необходимо отсканировать архивы рукописей, выполненных рельефно-точечным шрифтом.
	\item При инклюзивном образовании, когда незрячий ребёнок посещает общеобразовательную школу, даже специализированным педагогам сложно читать большое количество брайлевской литературы.
    \item В условиях дистанционной работы незрячий педагог должен проверять работы своего ученика.
\end{enumerate}

Особое значение программы распознавания шрифта Брайля имеют для организации учебного процесса.
По ряду причин (низкая пропускная способность специализированных образовательных учреждений, трудности в социализации и др.) доля незрячих, грамотных по Брайлю, в последние два десятилетия падает как в России, так и в других странах.
В помощь незрячим ученикам и педагогам в СПбПУ разрабатываются программно-аппаратные комплексы \cite{begun2019} \cite{zuev2021}, которые призваны ускорить освоение и закрепление материала.
В частности, разработана программа, распознающая по фотоснимкам рельефные карточки, на которых изображены увеличенные символы азбуки Брайля \cite{zuev2021}; планируется добавить в эту же программу распознавание брайлевского текста по фотографиям бумажных носителей.
\end{document}