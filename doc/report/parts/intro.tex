\documentclass{main.tex}[subfiles]
\begin{document}
\nonumsection{Введение}
По оценкам Всемирной организации здравоохранения, в мире более 285 миллионов незрячих или слабовидящих, и их число возрастает \cite{who_report2012}.
В XIX веке люди с ограничениями по зрению начали использовать для чтения и письма рельефно-точечный шрифт Л. Брайля; позже появились особые печатные машинки, а затем и рельефные дисплеи, поддерживающие алфавит Брайля.
В настоящее время обучение технике чтения и письма рельефно-точечным шрифтом входит в программу реабилитации инвалидов по зрению во всём мире.

Ожидается, что развитию и повышению удобства использования азбуки Брайля будет способствовать появление программ, распознающих рельефный шрифт по фотографии.
Подобные программы могут применяться, например, в следующих ситуациях:
\begin{enumerate}[noitemsep]
    \item Необходимо отсканировать архивы рукописей, выполненных рельефно-точечным шрифтом.
	\item При инклюзивном образовании, когда незрячий ребёнок посещает общеобразовательную школу, даже специализированным педагогам сложно читать большое количество брайлевской литературы.
    \item В условиях дистанционной работы незрячий педагог должен проверять работы своего ученика.
\end{enumerate}

Особое значение программы распознавания шрифта Брайля имеют для организации учебного процесса.
По ряду причин (низкая пропускная способность специализированных образовательных учреждений, трудности в социализации и др.) доля незрячих, грамотных по Брайлю, в последние два десятилетия падает как в России, так и в других странах.
В помощь незрячим ученикам и педагогам в СПбПУ разрабатываются программно-аппаратные комплексы \cite{begun2019, zuev2021}, которые призваны ускорить освоение и закрепление материала.
В частности, разработана программа, распознающая по фотоснимкам рельефные карточки, на которых изображены увеличенные символы азбуки Брайля \cite{zuev2021}; планируется добавить в эту же программу распознавание брайлевского текста по фотографиям бумажных носителей. \\

Разработка систем распознавания шрифта Брайля в последние годы активно обсуждается на разных уровнях.
Веб-сервис "Angelina Braille Reader"\hspace{0pt}, конвертирующий брайлевский шрифт на изображениях в компьютерный текст, получает информационную и грантовую поддержку со стороны Агенства стратегических инициатив и администрации президента России.

Впрочем, не все эксперты, работающие с незрячими, считают их нужными и полезными.
Так, на круглом столе в Общественной палате Российской Федерации в феврале 2021 года директор издательско-полиграфического объединения <<Чтение>> Всероссийского общества слепых Олег Николаевич Пилюгин высказал ряд тезисов против разработки и внедрения подобных программ:
\begin{enumerate}[noitemsep]
    \item Подобные разработки могут привести к тому, что уменьшится число педагогов, которые знают азбуку Брайля.
    \item Программы, распознающие текст и предназначенные для незрячих, произносят его вслух.
    Шрифт Брайля может изображать курсив, заглавные буквы и прочие элементы форматирования, которые теряются при озвучке.
    Нельзя подменять чтение по Брайлю прослушиванием аудио (как нельзя заменять чтение книг просмотром фильмов).
    \item В России выделяется недостаточно средств на закупку специализированного типографского оборудования для брайлевской печати и дисплеев Брайля.
    Бессмысленно тратить деньги на разработку программ распознавания бумажных брайлевских текстов, пока учреждения реабилитации инвалидов по зрению не оснащены в достаточной мере бумажными пособиями.
\end{enumerate}

Принимая во внимание вышеуказанные опасения, стоит, однако же, отметить, что появление в 2020 году сервиса "Angelina Braille Reader"\hspace{0pt} был с воодушевлением встречено педагогами и родителями учеников с нарушениями зрения.
К настоящему моменту на сайте\footnote{\href{http://angelina-reader.ru}{angelina-reader.ru}} зарегистрировано более 40 тысяч пользователей.

\end{document}