\documentclass{main.tex}[subfiles]
\begin{document}
\newpage
\nonumsection{Заключение}
В работе исследуется задача распознавания шрифта Брайля, проводится обзор разработанных ранее алгоритмов и предлагается схема улучшения программы Angelina Reader.
Достигнуты следующие результаты:
\begin{enumerate}[noitemsep]
    \item Собран набор данных, содержащий 1731 фотографию листов из книг, напечатанных шрифтом Брайля (суммарно более 1.5 миллиона знаков).
    \item Произведена автоматическая разметка (аннотация) страниц с помощью программы Angelina Reader \cite{ovodov2020}, предварительно обученной на выборке меньшего объёма.
    \item Разработан алгоритм автоматического исправления меток, полученных в процессе разметки, путём построения выравнивания между распознанным текстом и исходным текстом документа.
    \item Написана программа, реализующая вышеупомянутый алгоритм, которая затем применена к собранным данным.
    В результате исправления доля распознанных слов, присутствующих в оригинале текста, повышается в среднем с $ 0.838 $ до $ 0.996 $.
    \item Проведено обучение нейронной сети, лежащей в основе программы Angelina Reader, с использованием автоматически размеченных данных как обучающих (полуконтролируемое обучение).
    Показано, что при таком подходе улучшается качество распознавания страниц книг, сфотографированных на камеру, но ухудшается качество распознавания фотографий рукописей и сканов книг.
\end{enumerate}

Ещё одно потенциальное применение накопленного по ходу работы материала -- построенные выравнивания можно использовать для статистического анализа ошибок нейросети, чтобы на основе полученной информации разрабатывать алгоритмы постобработки распознанного текста в случае, когда оригинальный текст неизвестен.
Подобные исследования ранее проводились для обычного текста \cite{nguyen2019}.

Дальнейшим развитием этой работы может стать анализ и устранение по такой же схеме ошибок распознавания специализированной литературы, напечатанной шрифтом Брайля: математической, физической, химической, музыкальной и так далее.
\end{document}